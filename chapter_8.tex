\section{Conditioning and Disintegration}
In studying stochastic processes $(\xi_\alpha)_{\alpha \in A}$, we want to see how different random variables are related to each other. In particular, we want to see if observing one random variable will give more information on other random variables. For this, we need the notion of conditional probability and expectation.

\subsection{Conditional Probability}
\subsubsection{The Discrete Case}
Let $(\Omega, \F, \p)$ be a probability space. We recall the notions of conditional probability on an event: 

\begin{definition}[Conditional Probabilities]
Let $B \in \F$ be an event with $\p(B) > 0$. The \textit{conditional probability of $A$ with respect to $B$} is 
\begin{equation}
    \p(A \,|\, B) = \frac{\p(A \cap B)}{\p(B)}, \quad A\in \F.
\end{equation}
By convention, if $\p(B) = 0$ then we set $\p(A \,|\, B) = 0$
\end{definition}

\begin{exercise}
Check that $\p(\cdot | B)$ is a measure on $\F$.
\end{exercise}

Conditioning should be relative to information one has and $\sigma$-algebra are the natural carriers or descriptions for information content. We would thus like to condition on a $\sigma$-algebra. In the example above, we could replace $B$ by its complement $B^c$ and obtain a new measure $\p(A|B^c)$. Now, for any $\omega \in \Omega$, we have either $\omega \in B$ or $\omega \in B^c$ so it is natural to define
\begin{equation}
    \p(A | \sigma(B))(\omega) := \p(A|B)\chi_B(\omega) + \p(A | B^c)\chi_{B^c}(\omega).
\end{equation}
In this way, for a fixed $\omega \in \Omega$, $\p(\cdot | \sigma(B))(\omega)$ is a probability measure but for a fixed $A\in \F$, $\p(A|\sigma(B))(\cdot)$ is a random variable (taking two values).\\

The sets $B$ and $B^c$ partition $\Omega$, and these ideas carry over to the general partition. 
\begin{definition}[Conditional Expectation on $\sigma$-algebra generated by partition]
    Let $D = \{D_1, D_2, \dots \}$ be a finite or countable partition of $\Omega$ and let $\G = \sigma(D)$. For $A \in \F$ consider the function with values
    \begin{equation} \label{eq:conditional_probability_partition}
        f(\omega) := \p(A | D_i) = \frac{\p(A\cup D_i)}{\p(D_i)}, \quad \text{if }\omega \in D_i.
    \end{equation}
    This function or random variable $f$ is called the \textbf{conditional probability of $A$ given $\G$} and is denoted $\p(A | \G)$. This is written as $\p(A | \G)_\omega$ or $[\p(A | \G)](\omega)$, whenever the argument $\omega$ needs to be explicitly shown.
\end{definition}

\begin{exercise}
Convince yourself that if $\G = \set{\emptyset, \Omega}$ then $f(\omega) := [\p(A\,|\,\G)](\omega)$ is the constant function $f(\omega) := \p(A)$.
\end{exercise}

If the observer learns which element $D_i$ of the partition it is that contains $\omega$, then his new probability for the event $\omega \in A$ is $f(\omega)$. The partition $\{D_i\}$, or equivalently the $\sigma$-algebra, $\G$, can be regarded as an experiment, and to learn which $D_i$ it is that contains $\omega$ is to learn the outcome of the experiment. Thus $\p(A | \G)$ is a function whose value on $D_i$ is the ordinary conditional probability $\p(A | D_i)$. This definition needs to be completed, because $\p(A | D_i)$ is not defined if $\p(D_i) = 0$. In this case $\p(A | \G)$ will be taken to have any constant value on $D_i$; the value is arbitrary but must be the same over all of the set $D_i$.
\begin{example}
Suppose that $X_0, X_1, \dots$ is a Markov chain with state space $S$. The events
\begin{equation*}
    \{ X_0 = i_0,\dots, X_n = i_n \}, \quad i_0,\dots, i_n \in S
\end{equation*}
form a finite or countable partition of $\Omega$. If $\G_n$ is the $\sigma$-algebra generated by this partition, then by the defining condition for Markov chains, 
\begin{equation*}
    \p(X_{n+1} = j | \G_n)_\omega = \p(X_{n+1} = j|X_n = i_n),
\end{equation*}
for $\omega \in \{ X_0 = i_0,\dots, X_n = i_n \}$, $i_0,\dots, i_n \in S$. The sets $\{X_n = i \}$ for $i\in S$ also partition $\Omega$, and they generate a $\sigma$-algebra $\G_n^0$ smaller than $\G_n$. Now 
\begin{equation*}
    \p(X_{n+1} = j | \G_n^0)_\omega = \p(X_{n+1} | X_n = i),
\end{equation*}
for $\omega \in \{X_n = i\}, i \in S$, and the essence of the Markov property is that 
\begin{equation*}
    \p(X_{n+1} | \G_n) = \p(X_{n+1}|\G_n^0).
\end{equation*}
\end{example} 
\subsubsection{The General Case}
If $\G$ is the $\sigma$-algebra generated by a partition $D_1, D_2, \dots$, then the general element of $\G$ is a disjoint union $D_{i_1} \cup \D_{i_2} \cup \cdots$, finite or countable, of certain of the $D_i$. To know which set $D_i$ it is that contains $\omega$ is the same thing as to know which sets in $\G$ contain $\omega$ and which do not. This second way of looking at the matter carries over to the general $\sigma-$algebra $\G$ contained in $\F$ (as always, we have the probability space $(\Omega, \F, \p)$). The $\sigma-$algebra $\G$ will not in general come from a partition as above.\\

One can imagine an observer who knows for each $G$ in $\G$ whether $\omega \in G$ or $\omega \in G^c$. Thus the $\sigma$-algebra $\G$ can in principle be identified with an experiment or observation. Is is natural to try and define conditional probabilities $\p(A | \G)$ with respect to the experiment $\G$. To do this, fix an $\A$ in $\F$ and define a finite measure $\nu$ on $\G$ by
\begin{equation}
    \nu (G) = \p(A \cap G), \quad G \in \G.
\end{equation}
Then $\p(G) = 0$ implies that $\nu (G) = 0$. The Radon-Nikodym theorem can be applied to the measures $\nu$ and $\p$ on the measurable space $(\Omega, \G)$ because the first one is absolutely continuous with respect to the second. It follows that there exists a function or a random variable $f$, which is $\G-$measurable and integrable with respect to $\p$, such that 
\begin{equation}
    \p(A \cap G) = \nu (G) = \int_G f \d \p,
\end{equation}
for all $G$ in $\G$. Denote this function $f$ by $\p(A | \G)$. It is a random variable with two properties:
\begin{enumerate}[(i)]
    \item $\p(A | \G)$ is $\G$-measurable and integrable
    \item $\p(A | \G)$ satisfies the functional equation
    \begin{equation} \label{eq:conditional_probability_sigma_algebra}
        \int_G \p(A | \G) \d \p = \p(A \cap G), \quad G \in \G.
    \end{equation}
\end{enumerate}
There will in general be many such random variables $\p(A | \G)$, but any two of them are equal with probability $1$.\\

If $\G$ is generated by a partition $D_1, D_2, \dots$, the function $f$ defined by $(8.1)$ is $\G$-measurable because $\{ \omega: f(\omega) \in H \}$ is the union of those $D_i$ over which the constant value of $f$ lies in $H$. Any $G$ in $\G$ is a disjoint union $G = \bigcup_{k} D_{i_k}$, and 
\begin{equation}
    \p(A \cap G) = \sum_k \p(A | D_{i_k})\p(D_{i_k}),
\end{equation}
so that \eqref{eq:conditional_probability_partition} satisfies \eqref{eq:conditional_probability_sigma_algebra} as well. Thus the general definition is an extension of the one for the discrete case.\\

Condition $(i)$ above requires that the values of $\p(A|\G)$ depend only on the sets in $\G$. An observer who knows the outcome of $\G$ viewed as an experiment knows for each $G$ in $\G$ whether it contains $\omega$ or not. For each $x$ he knows this in particular for the set $\{\omega' : \p(A|\G)_{\omega'} = x\}$, and hence he knows in principle the functional value $\p(A|\G)_\omega$ even if he does not know $\omega$ itself.\\

Condition $(ii)$ has a gambling interpretation. Suppose that the observer, after he has learned the outcome of $\G$ is offered the opportunity to bet on the event $A$ (unless $A$ lies in $\G$, he does not yet know whether or not it occurred). He is required to pay an entry fee of $\p(A|\G)$ units and will win $1$ unit if $A$ occurs and nothing otherwise. If the observer decides to bet and pays the fee, he gains $1 - \p(A|\G)$ if $A$ occurs and $-\p(A | \G$ otherwise, so that his gain is 
\begin{equation*}
    (1 - \p(A | \G)) \chi_A + (-\p(A|\G))\chi_{A^c} = \chi_A - \p(A | \G).
\end{equation*}
If he declines to bet, his gain is of course $0$. Suppose that he adopts the strategy of betting if $G$ occurs but not otherwise, where $G$ is some set in $\G$. He can actually carry out this strategy, since after learning the outcome of the experiment $\G$ he knows whether or not $G$ occurred. His expected gain with this strategy is his gin integrated over $G$:
\begin{equation*}
    \int_G (\chi_A - \p(A | \G)) \d \p.
\end{equation*}
But \eqref{eq:conditional_probability_sigma_algebra} is exactly the requirement that this vanish for each $G$ in $\G$. Condition $(ii)$ requires then that each strategy be fair in the sense that the observer stands neither to win nor to lose on the average. Thus $\p(A | \G)$ is just entry fee, as intuition requires.\\


\subsection{Conditional Expectation}
We now develop the theory of condition expectation from first principles. 

\begin{definition}[Conditional expectation] \label{def:conditional_expectation}
Suppose that $\xi$ is an integrable random variable on $(\Omega, \F, \p)$ and that $\G$ is a $\sigma-$algebra contained in $\F$. There exists a random variable $\E[\xi | \G]$, called \textbf{conditional expectation of $\xi$ given $\G$}, having these two properties:
\begin{itemize}
    \item $\E[\xi | \G]$ is $\G$-measurable and integrable.
    \item For every $G \in \G$,
    \begin{equation} \label{eq:condtional_expectation}
        \E[\chi_G \E[\xi | \G]] = \int_G \E[\xi | \G] \d \p = \int_G \xi \d \p = \E[\chi_G \xi]
    \end{equation}
\end{itemize}
\end{definition}

\begin{theorem}[Existence of conditional expectation] Whenever $\xi$ is an integrabl random variable on $(\Omega, \F, \p)$, and that $\G$ is a $\sigma$-algebra contained in $\F$, then $\E[\xi \,|\, \G]$ exists, and is unique almost everywhere.
\end{theorem}

To prove the existence of such a random variable, we recall the following generalised version of the Randon-Nikodym theorem.

\begin{theorem}[Radon-Nikodym Theorem]
Let $(\Omega, \F)$ be a measure space, $\mu$ - a finite measure on $\F$. Let $\lambda$ be a measure on $\F$ a.c. with respect to $\mu$ (i.e. $\lambda (A) = 0$ whenever $\mu(A)=0)$. Then there exists an $\F$-measurable function $f$ such that 
\begin{equation}
    \lambda(A) = \int_A f \d \mu \quad \forall A \in \F.
\end{equation}
This function is determined uniquely, up to sets of measure zero. It is called the derivative of $\lambda$ w.r.t. $\mu: f = \frac{\d \lambda}{\d \mu}$. 
\end{theorem}

Consider first the case of nonnegative $\xi$. Define a measure $\nu$ on $\G$ by
\begin{equation*}
    \nu(G) = \int_G \xi \, \d \p.
\end{equation*}
This measure is finite because $\xi$ is integrable, and it is absolutely continuous with respect to $\p$. By the Radon-Nikodym theorem there is a function $f$, $\G-$measurable, such that $\nu(G) = \int_G f \d \p$. This $f$ has the two properties of our definition. If $\xi$ is not necessarily nonnegative, $\E[\xi^+ | \G] - \R[\xi^- | \G]$ clearly has the required properties. \\

There will in general be many such random variables $\E[\xi | \G]$. Any one of them is called a version of the conditional expected value. Any two versions are equal with probability $1$.

\begin{example}[Conditioning on $\sigma$-algbera generated by partition]
Suppose that $G_1, G_2, \dots$ is a finite or countable partition of $\Omega$ generating the $\sigma-$algebra $\G$. Then $\E[\xi | \G]$ must, since it is $\G-$measurable, have some constant value over $G_i$, say $\alpha_i$. Then
\begin{align}
    \int_{G_i} \E[\xi | \G] \d \p &= \int_{G_i} \alpha_i \, \d \p = \alpha_i \p(G_i).
\end{align}
Therefore,
\begin{equation}
    \alpha_i \p (G_i) = \int_{G_i} \xi \d \p,
\end{equation}
which implies
\begin{equation}
    \E[\xi | \G]_\omega = \frac{1}{\p(G_i)} \int_{G_i} \xi \d \p, \quad \omega \in \G_i.
\end{equation}
If $\p(G_i) = 0$, then the value of $\E[\xi | \G]$ over $G_i$ is constant but arbitrary.
\end{example}

To familiarise yourself with the definition of conditional expectation on a $\sigma$-algebra, we provide the following exercise.
\begin{exercise} \label{ex:tower_property}
\begin{enumerate}
    \item[]
    \item Convince yourself that if $\G = \set{\emptyset, \Omega}$ then $[\E[\xi\,|\,\G]](\omega)$ is equal to the constant function $f(\omega) := \E(\xi)$ almost everywhere.
    \item More challenging: let's say you know that $\xi$ is independent of $\G$, which means that for all $B \in \G$ we have $\xi$ independent of $\chi_B$, then we still have $\E[\xi\,|\,\G] = \E[\xi]$
    \item Convince yourself that $\E[\xi \,|\, \F] = \xi$ almost everywhere.
    \item Show that $\E[\E[\xi \,|\, \G]] = \E[\xi]$.
\end{enumerate}
\end{exercise}

\begin{hint}
All problems can be solved solely by checking the definition. For example, in question 1 you check that equality \eqref{eq:conditional_probability_sigma_algebra} holds for $G = \phi$ and $G = \Omega$, and in question 4 you directly use the equality \eqref{eq:conditional_probability_sigma_algebra} for appropriate choice of $G \in \G$. You should also try to convince yourself that the above results are intuitive, e.g. in question 3 the conditional expectation is the random variable itself since you know everything about the random variable.
\end{hint}

\begin{solution} (For question 2):
This is perhaps the most tricky question, but again we try to resort to the definition \eqref{eq:conditional_probability_sigma_algebra}. Here we want to show for all $G \in \G$
\begin{equation}
    \E[\xi] \E[\chi_G] \E[\chi_G \E[\xi]] = \E[\chi_G \xi]
\end{equation}
which is true by the independence assumption. This makes sense, because the $\sigma$ algebra does not give additional information about the random variable $\xi$.
\end{solution}

The example below links conditional expectation and conditional probabilities.

\begin{example}[Conditional expectation of indicator function]
For an indicator random variable $\chi_A$ the defining properties of $\E[\chi_A | \G]$ and $\p(A| \G)$ coincide, therefore 
\begin{equation}
    \E[\chi_A | \G] = \p(A | \G)
\end{equation}
almost surely. For a simple random variable $\xi = \sum \alpha_i \chi_{A_i}$,
\begin{equation}
    \E[\xi | \G] = \sum \alpha_i \p(A_i | \G)
\end{equation}
almost surely.
\end{example}

\subsection{Properties of conditional expectation}
We list some of the properties for conditional expectation. Most properties are direct applications of the definition \eqref{eq:conditional_probability_sigma_algebra}, and we encourage you to prove them by yourselves. The proofs are, however, still included if you want extra hint to familiarise yourself with the notion of conditional expectation.

\begin{property}[I. Linearity] 
Let $\xi,\eta$ is a random variable on $(\Omega, \F, \p)$ and let $\G$ be a $\sigma-$algebra contained in $\F$. Then, almost surely,
\begin{equation} \label{eq:linearity_conditional_expectation}
    \E[a\xi + b\eta + c \,|\, \G] = a\E[\xi \,|\, \G] + b \E[\eta \,|\, \G ] + c
\end{equation}
\end{property}

\begin{proof}
For all $G \in \G$, we have
\begin{align*}
    \E[\chi_G \E[a\xi + b\eta + c \,|\, \G]] 
    &= \E[\chi_G (a\xi + b\eta + c)] \\
    &= a\E[\chi_G\xi] + b\E[\chi_G \eta] + c\E[\chi_G] \\
    &= a\E[\chi_G \E[\xi\,|\,\G]] + b\E[\chi_G \E[\eta\,|\,\G]] + c\E[\chi_G] \\
    &= \E[\chi_G (a\E[\xi \,|\, \G] + b \E[\eta \,|\, \G ] + c)]
\end{align*}
\end{proof}

\begin{property}[II. Monotonicity] \label{prop:conditional_expectation_monotonicity} 
Let $\xi, \eta$ is a random variable on $(\Omega, \F, \p)$ such that $\xi \leq \eta$ almost surely, and let $\G$ be a $\sigma-$algebra contained in $\F$. Then, almost surely, $\E[\xi\,|\,\G] \leq \E[\eta \,|\, \G]$. The above result also holds when $\geq$ is replaced by $>$.
\end{property}

\begin{proof}
Consider the set $G = \set{\omega\,|\,\E[\xi\,|\,\G] > \E[\eta\,|\,\G]}$. If $\p(G) > 0$ then $\E[\chi_G (\E[\xi\,|\,\G] - \E[\eta\,|\,\G])] > 0$ by our definition of $G$. But we also know that $\E[\chi_G (\E[\xi\,|\,\G] - \E[\eta\,|\,\G])] = \E[\chi_G (\xi-\eta)] \leq 0$, which is a contradiction. So $\p(G) = 0$, and $\E[\chi_G (\E[\xi\,|\,\G] - \E[\eta\,|\,\G])] = 0$. So $\E[(\E[\xi\,|\,\G] - \E[\eta\,|\,\G])] = \E[\chi_{G^C} (\E[\xi\,|\,\G] - \E[\eta\,|\,\G])] \leq 0$, and hence $\E[\xi\,|\,\G] \leq \E[\eta\,|\,\G]$ almost surely.
\end{proof}

Substituting $\xi = 0$, then we know that $\eta \geq 0$ a.s. $\implies$ $\E[\eta\,|\,\G] \geq$ a.s.

\begin{exercise}
Prove that for all integrable $\xi$, we have $|\E[\xi|\G]| \leq \E[|\xi| \, | \G]$.
\end{exercise}

\begin{hint}
Note that $\xi = \xi^+ - \xi^-$ and utilise triangle inequality. Also note that $|\xi| = \xi^+ + \xi^-$.
\end{hint}

We can hence prove a conditional version of Jensen inequality:
\begin{theorem}[Conditional Jensen's Inequality]
Suppose that $(\Omega, \F,\p)$ is a probability space and $\xi$ is an integrable random variable taking values in an open interval $I \subset \R$. Let $g: I \to \R$ be convex and let $\G$ be a sub $\sigma-$algebra of $\F$. If $\E[|g(\xi)|] < \infty$, then
\begin{equation} \label{eq:conditional_Jensen_ineq}
    \E[\varphi(\xi) | \G] \ge \varphi(\E[\xi | \G])  \text{ almost surely.}
\end{equation}
\end{theorem}

\begin{property}[III. Tower/Telescopic Properties] \label{prop:tower_property_general}
Let $(\Omega, \F,\p)$ be a probability space, $\xi$ an integrable random variable and $\F_1, \F_2$ be $\sigma-$algebras with $\F_1 \subseteq \F_2 \subseteq \F$. Then almost surely
\begin{equation}
\E[\E[\xi | \F_2] | \F_1] = \E[\xi | \F_1] = \E[\E[\xi | \F_1] | \F_2]
\end{equation}
\end{property}

\begin{hint}
This is a generalisation of exercise \ref{ex:tower_property}.
\end{hint}

\begin{proof}
The first equality can be proven by just using the definition: consider arbitrary $G \in \F_1$, then $G \in \F_2$ and hence 
\begin{equation}
    \E[\chi_G \E[\xi | \F_2]] = \E[\chi_G \xi] = \E[\chi_G \E[\xi | \F_1]]
\end{equation}
which establish the first equality. The second equality can be proven by using similar method in  question 3 from exercise \ref{ex:tower_property}, noting $\E[\xi\,|\,\F_1]$ is $\F_1$ measurable then it is also $\F_2$ measurable.
\end{proof}

\begin{property}[IV. On Taking Limits Under the Conditional Expectation Sign]
Let $\{ \xi_n \}_{n \ge 1}$ be a sequence of extended random variables.
\begin{enumerate}
    \item (Dominated convergence) If $|\xi_n| \le \eta$, $\E[\eta]<\infty$, and $\xi_n \to \xi$ $(a.s.)$, then
    \begin{equation*}
        \E[\xi_n \,|\, \G] \xrightarrow{a.s.} \E[\xi \,|\, \G]
    \end{equation*}
    and 
    \begin{equation*}
        \E[|\xi_n - \xi| | \G] \xrightarrow{a.s.} 0.
    \end{equation*}
    \item (Monotone convergence) If $\xi_n \ge \eta$, $\E[\eta] > -\infty$, $\xi_n \uparrow \xi \, (a.s.)$ and $\E[|\xi|] < \infty$, then
    \begin{equation*}
        \E[\xi_n | \G] \uparrow \E[\xi | \G] \quad (a.s.)
    \end{equation*}
    \item If $\xi_n \le \eta$, $\E[\eta] < \infty$, and $\xi_n \downarrow \xi \, (a.s.)$, then
    \begin{equation*}
        \E[\xi_n | \G] \downarrow \E[\xi | \G] \quad (a.s.)
    \end{equation*}
    \item (Fatou) If $\xi_n \ge \eta $, $\E[\eta] > - \infty$, then
    \begin{equation*}
        \E[\liminf \xi_n | \G] \le \liminf \E[\xi_n | \G] \quad (a.s.)
    \end{equation*}
    \item If $\xi_n \le \eta $, $\E[\eta] < \infty$, then
    \begin{equation*}
        \E[\limsup \xi_n | \G] \le \limsup \E[\xi_n | \G] \quad (a.s.)
    \end{equation*}
    \item (Summation) If $\xi_n \ge 0$ then
    \begin{equation*}
        \E\big[\sum \xi_n | \G \big] = \sum \E[\xi_n | \G] \quad (a.s.)
    \end{equation*}
\end{enumerate}
\end{property}

\begin{proof}
\begin{enumerate}
\item[]
\item Let $\zeta_n = \sup_{m\geq n} |\xi_m - \xi|$. Then $0 \leq |\zeta_n| \leq 2\eta$ and $\zeta_n \to 0$ almost surely, so by DCT we have $\E[\zeta_n] \overset{n\to\infty}\to 0$ Now by triangle inequality, one has 
\begin{equation*}
0 \leq \abs{\E[X_n|\G] - \E[X|\G]} \leq \E[|X_n-X| \,|\, \G] \leq \E[\zeta_n|\G].
\end{equation*}
Since the sequence $\E[\zeta_n|\G](\omega)$ is decreasing on $n$ for fixed $\omega$, the sequence $\E[\zeta_n|\G](\omega)$ exhibits limits $\omega$-almost surely. To evaluate the limit, notice that
\begin{equation*}
0 \leq \E\sqbracket{\lim_{n\to\infty} \E[\zeta_n|\G]} \leq \E\sqbracket{\E[\zeta_n|\G]} = \E[\zeta_n] \overset{n\to\infty}\to 0,
\end{equation*}
so $\lim_{n\to\infty} \E[\zeta_n|\G] = 0$ almost surely, completing the proof.
\item We note from the proof of MCT that one can set $\eta = 0$. Let $\tilde{\xi}_n = \E[\xi_n|\G]$. Then by property II (monotonicity) we have $\tilde{\xi}_n \geq 0$ almost surely and the events $A_n = \set{\tilde{\xi}_n < \tilde{\xi}_{n-1}} \in \G$ has $\p(A_n) = 0$. Let $\tilde{\xi} := \limsup_{n\to\infty} \tilde{\xi}_n$ and $A = \cup_{n\geq 2}$. Then $A \in \G$, $\p(A) = 0$ and for all $\omega \in A^c$ we know that $\tilde{\xi}_n(\omega) \nearrow \tilde{\xi}(\omega)$. We evaluate the limits by noting that for all $G \in \G$,
\begin{equation*}
    \E\sqbracket{\tilde{\xi} \chi_G} = \E\sqbracket{\tilde{\xi}\chi_{G \cap A^c}} \overset{\text{MCT}}= \lim_{n\to\infty} \E\sqbracket{\tilde{\xi}_n \chi_{G \cap A^c}} = \lim_{n\to\infty} \E\sqbracket{\xi \chi_{G \cap A^c}} \overset{\text{MCT}} = \E\sqbracket{\xi\chi_{G \cap A^c}} = \E\sqbracket{\xi\chi_{G}},
\end{equation*}
so we see that $\tilde{\xi}$ is integrable (by taking $G \in \G$) and that $\xi = \tilde{\xi}$ almost surely as desired.
\item This is trivially equivalent to (2) by considering the sequence $\xi - \xi_n$.
\item This is a direct application to (2) by considering the sequence $\eta_n = \inf_{k\geq n} \xi_n$, which is increasing.
\item This is trivially equivalent to (4).
\item This is a direct application to (2).
\end{enumerate}
\end{proof}

\begin{corollary}[V. Factorisation] \label{cor:conditional_expectation_factorisation}
Let $\xi$ and $\eta$ be random variables on $(\Omega, \F, \p)$ with $\xi$, $\eta$ and $\xi \eta $ integrable. Let $\G \subset \F$ be  a $\sigma$-algebra and suppose that $\eta$ is $\G-$measurable. Then
\begin{equation}
	\E[\xi \eta | \G] = \eta \E[\xi | \G] \text{ almost everywhere}
\end{equation}
\end{corollary}

\begin{hint}
Sorry this is not an easy proof. We utilise the four step proof, see chapter 1.
\end{hint}

We finally have the following
\begin{proposition}[$L^2$ orthogonal projection] \label{prop:conditional_expectation_proj}
Suppose $\xi$ is a random variable on $(\Omega,\F,\p)$ such that $\E[\xi^2] < \infty$ (i.e. $\xi \in L^2$). Then $\E[X\,|\,\G]$ is the $L^2$ orthogonal projection onto the subspace 
\begin{equation}
\mathcal{P} = \set{g \in L^2 \,|\, g \text{ is }\G\text{ measurable.}}
\end{equation}
In other words, $\E[X\,|\,\G]$ is the unique minimiser of $\E[X-Y]^2$ for $Y \in \G$.
\end{proposition}

\begin{proof}
We first note that $\E[\E[X\,|\,\G]^2] \leq \E[X^2]$ by conditional Jensen inequality \eqref{eq:conditional_Jensen_ineq}, and that $\E[X\,|\,\G]$ is $\G$-measurable, so $\E[X\,|\,\G] \in \G$. Let $Z$ be a $\G$-measurable function, then \begin{align*}
    \E[X-Z]^2 &= \E[X-\E[X\,|\,\G]+\E[X\,|\,\G]-Z]^2 \\
    &= \E[X-\E[X\,|\,\G]^2 + \E[X\,|\,\G]-Z]^2 + 2\E[(X-\E[X\,|\,\G])(\E[X\,|\,\G]-Z)]
\end{align*}
It remains to show that the cross term is zero. Notice that $\E[X\,|\,\G]-Z$ is $\G$-measurable, so by tower property (exercise \ref{ex:tower_property})
\begin{align*}
    \E[(X-\E[X\,|\,\G])(\E[X\,|\,\G]-Z)] &= \E[\E[(X-\E[X\,|\,\G])(\E[X\,|\,\G]-Z)\,|\, \G]] \\
    &= \E[(\E[X\,|\,\G]-\E[X\,|\,\G])(\E[X\,|\,\G]-Z)] = 0
\end{align*}
the last equality is justified by the factorisation property (corollary \ref{cor:conditional_expectation_factorisation}), linearity and another application of tower property. So we have
\begin{equation}
    \E[X-Z]^2 \geq \E[X-\E[X\,|\,\G]^2 + \E[X\,|\,\G]-Z]^2 \geq \E[X-\E[X\,|\,\G]^2
\end{equation}
with equality holds if $\E[X\,|\,\G]=Z$ almost everywhere.
\end{proof}

\subsection{Conditioning on a random variable}
We define the following
\begin{definition}
The \textit{conditional expectation} of a random variable $\xi$ with respect to a random variable $\eta$ is defined as follows
\begin{equation*}
    \E[\xi | \eta] \equiv \E[\xi | \sigma(\eta)],
\end{equation*}
where $\sigma(\eta)$ is the $\sigma-$algebra generated by $\eta$.
\end{definition}

\begin{theorem}[Representation of conditional expectation] \label{thm:rep_of_conditional_expectation}
There exists a unique (almost everywhere) Borel function $g: \R \to \R$ such that
\begin{equation}
    \E[\xi | \eta] = g(\eta).
\end{equation}
\end{theorem}

The proof is a direct application of the following lemma from measure theory

\begin{lemma}
Let $\mu, \eta$ be random variables on $(\Omega,\F,\p)$. Then $\mu$ be $\sigma(\eta)$-measurable $\iff$ there exists a Borel-measurable function $f: \R \to \R$ such that $\mu = f(\eta)$. 
\end{lemma}

\begin{hint}
We apply four-step proof on $\mu$.
\end{hint}

\begin{proof}
\begin{enumerate}
    \item[]
    \item Let $\mu = \chi_A$ for some $A \in \F$. For $\mu$ to be $\sigma(\eta)$ measurable we must have $A \in \sigma(\eta)$. (Why?) This means there exists $B \in \B(\R)$ such that $\eta^{-1}(B) = A$. We immediate see that $\mu = \chi_B(\eta)$.
    \item Let $\mu$ be a simple random variable such that $\mu = \sum_{i=1}^n c_j \chi_{A_j}$, with $\set{A_j}_{j=1}^n$ partitions $\Omega$. Again we must have $A_j \in \sigma(\eta)$, so for all $j$ there exists $B \in \B(\R)$ such that $\eta^{-1}(B_j) = A_j$. Then $\set{B_j}_{j=1}^n$ partitions $\eta(\Omega)$, and that $\mu = f(\eta)$ with $f = \sum_{i=1}^n c_j \chi_{B_j}(x)$.
    \item We assume $\mu$ being non-negative, then $\mu$ can be approximated by a pointwise non-decreasing sequence of simple random variable which converges pointwise to $\mu$: $\mu_n \nearrow \mu$. Each $\mu_n$ can be represented in the form of $f_n(\eta)$. Choose $f = \sup_{n\geq 1} f_n$, which is Borel measurable, and for all $\omega$, 
    \begin{equation}
        f(\eta(\omega)) = \sup_{n\geq 1} \mu_n(\omega) = \mu(\omega)
    \end{equation}
    \item For general $\mu$, we decompose it into $\mu^+ - \mu^-$, and write $\mu^+ = f^+(\eta)$ and $\mu^- = f^-(\eta)$ by step 3. We let $f = f^+ - f^-$ to complete the proof.
\end{enumerate}
\end{proof}

\begin{example}[Conditional expectation of random variables with joint density] \label{eg:conditional_expectation_joint_rv}
Consider real valued random variables $X,Y$ on same probability space $(\Omega,\F,\p)$. Assume the random vector $(X,Y)$ has continuous joint density $f_{X,Y}(x,y) > 0$. Recall that $X$ has density $f_X(x) = \int_\Omega f_{X,Y}(x,y) \, dy$ and $Y$ has density $f_Y(y) = \int_\Omega f_{X,Y}(x,y) \, dx$. Assume $f_X(x), f_Y(y) > 0$ almost everywhere in $\R$. We want to compute $\E[h(X)\,|\,Y]$ for all Borel-measurable function $h$ with $\E[|h(X)|] < \infty$. By theorem \ref{thm:rep_of_conditional_expectation}, we know that $\E[h(X)\,|\,Y] = \phi(Y)$ for a unique (almost everywhere) Borel-measurable $\phi$. We claim that
\begin{equation} \label{eq:conditional_expectation_rv}
    \phi: y \mapsto \int_\R h(x) \frac{f_{X,Y}(x,y)}{f_Y(y)} \, dx
\end{equation}
We can show this by utilising the definition of conditional expectation: for all $A \in \sigma(Y)$ we have
\begin{equation}
    \E[\chi_A \phi(Y)] = \E[\chi_A h(X)]
\end{equation}
Since $A \in \sigma(Y) \subseteq \F$, we know that $A = Y^{-1}(B)$ for some $B \in \B(\R)$.

\begin{align*}
    \E[\chi_A h(X)] &= \E[\chi_B(Y) h(X)] \\
    &= \int_\R \int_\R h(x) \chi_B(y) f_{X,Y}(x,y) \, dy \, dx \\
    &= \int_\R \int_\R h(x) \chi_B(y) \frac{f_{X,Y}(x,y)}{f_Y(y)} f_Y(y) \, dy \, dx \\
    &\overset{\text{(Tonelli)}}{=} \int_\R \int_\R h(x) \chi_B(y) \frac{f_{X,Y}(x,y)}{f_Y(y)} f_Y(y) \, dx \, dy \\
    &= \int_\R \chi_B(y) \bracket{\int_\R h(x) \frac{f_{X,Y}(x,y)}{f_Y(y)} \, dx} \, f_Y(y) dy \\
    &= \E[\chi_B(Y) \phi(Y)] \\
    &= \E[\chi_A \phi(Y)]
\end{align*}
which completes the proof. In particular, this shows us that if $C = X^{-1}(D)$
\begin{equation} \label{eq:regular_conditional_distribution_joint_rv}
    \p(X \in C \,|\, Y) = Q(Y;C), \; \text{with} \; Q(y;C) = \int_\R \chi_C(x) \frac{f_{X,Y}(x,y)}{f_Y(y)} \, dx
\end{equation}
This example can be generalised to the case when $X,Y$ are random vectors.
\end{example}

\begin{exercise}[Conditional expectation of discrete random variables] \label{ex:conditional_expectation_discrete_rv}
\begin{itemize}
    \item[]
    \item Consider random variables $X,Y$ taking value in $\N$, and assume they have joint mass $p_{X,Y}(x,y)$ for $x,y \in \N$. Assume $h:\N \to \R$ such that $\E[|h(X)|] < \infty$. Using \eqref{eq:conditional_probability_partition}, verify that $\E[h(X)\,|\,Y] = \phi(Y)$, where for $y$ such that $p_Y(y) \neq 0$,
    \begin{equation}
        \phi(y) = \sum_{x\in\N} h(x) \frac{p_X(x,y)}{p_Y(y)}
    \end{equation}
    Notice how this formula is similar to \eqref{eq:conditional_expectation_rv}. What value should we assign to $\phi(y)$ for $p_Y(y) = 0$ if we were to follow our convention about conditional probability on zero-probability events?
    \item Here is an application to the above formula. Consider random variables $Z_1, Z_2$ on $(\Omega,\F,\p)$ with $Z_1 \sim \Po(\lambda_1)$ and $Z_2 \sim \Po(\lambda_2)$. Assume $p = \lambda_1/(\lambda_1+\lambda_2)$, show that
    \begin{equation}
        \p[Z_1 = k \,|\, Z_1 + Z_2 = n] = {n \choose k} p^k (1-p)^{n-k}
    \end{equation}
\end{itemize}
\end{exercise}

We will study the above examples further in the next section. Before that, let us highlight the projection property of conditional expectation.

\begin{proposition}[$L^2$ orthogonal projection II]
If $\E[\xi^2] < \infty$, then 
\begin{equation*}
    \min_f \E[(\xi - f(\eta)^2] = \E[(\xi - \E[\xi | \eta])^2],
\end{equation*}
where min is over all $\sigma(\eta)$-measurable functions such that $\E[f^2(\eta)] < \infty$.
\end{proposition}

\begin{proof}
This is a direct application of proposition \ref{prop:conditional_expectation_proj}.
\end{proof}

We can therefore obtain conditional expectation by obtaining the minimiser of $L^2$ error.

\begin{exercise}[Conditional expectation of normal distribution] Consider random variables $X,Y$ such that they are jointly normally distributed:
\begin{equation}
    \begin{pmatrix} X \\ Y\end{pmatrix} \sim \mathbf{N}_{2} \bracket{ \begin{pmatrix} \mu_X \\ \mu_Y \end{pmatrix}, \begin{pmatrix} \sigma_X^2 & \rho \sigma_X \sigma_Y \\ \rho \sigma_X \sigma_Y & \sigma_Y^2 \end{pmatrix}},
\end{equation}

then $\E[Y|X] = f(X)$. Assume we know that $f(x) = ax+b$, then we know that $a,b$ are minimiser of $\E[Y - aX - b]^2$. Verify that
\begin{equation}
    \E(Y\,|\,X) = \mu_Y + \rho \frac{\sigma_Y}{\sigma_X}(X - \mu_X) = \bracket{\mu_Y - \rho \frac{\sigma_Y}{\sigma_X}\mu_X} + \rho \frac{\sigma_Y}{\sigma_X} X.
\end{equation}
\end{exercise}
\newpage

\subsection{Regular conditional distribution*}
\begin{unexaminable}
Let us revisit the above examples. We know from example \ref{eg:conditional_expectation_joint_rv} that if $(X,Y)$ has joint density $f_{X,Y}(x,y) > 0$ then we have
\begin{equation}
    \p(X \in C \,|\, Y) = Q(Y;C), \; \text{with} \; Q(y;C) = \int_\R \chi_C(x) \frac{f_{X,Y}(x,y)}{f_Y(y)} \, dx.
\end{equation}

There are two ways to interpret $Q(y;C)$. If we fix a set $C \in \F$, then the function $Q(.;C)$ is a $\sigma(Y)$-measurable. On the other hand, if we fix $y \in Y(\Omega)$, then the set function $Q(y;.)$ is a measure on $(\R,\B(\R))$. To see this it is trivial to see that $Q(y;\varnothing) = 0$ and 
\begin{equation}
    Q(y,\R) = \frac{1}{f_Y(y)} \int_\R f_{X,Y}(x,y) \, dx = 1 
\end{equation}

We only need to prove $\sigma$-additivity. Let $A_1, A_2, ...$ be disjoint sets in $\B(\R)$, and let $A = \sqcup_{i\geq 1} A_i$, then by monotone convergence theorem
\begin{align*}
    Q(y,A) = \int_\R \chi_A(x) \frac{f_{X,Y}(x,y)}{f_Y(y)} \, dx 
    &= \int_\R \sum_{i\geq 1} \chi_{A_i}(x) \frac{f_{X,Y}(x,y)}{f_Y(y)} \, dx \\ &\overset{\text{(MCT)}}{=} \sum_{i\geq1} \int_\R \chi_{A_i}(x) \frac{f_{X,Y}(x,y)}{f_Y(y)} \, dx = \sum_{i\geq 1} Q(y,A_i)
\end{align*}

$Q(y,C)$ is hence considered as a \textit{(Markov) stochastic kernel}. In this chapter, we would like to prove that we can construct such transitional kernel $Q_{Y,\G}(\omega, C) = [\p(Y \in C \,|\, \G)](\omega)$ for any random variables $Y$ and $\sigma$-algebra $\G$. We begin by formally defining the notion of transitional kernel

\begin{definition}[Transitional, Stochastic kernel]
Let $(\Omega_1, \F_1)$, $(\Omega_2,\F_2)$ be measurable spaces. A map $Q:\Omega_1 \times \F_2 \to [0,\infty]$ is a \textit{transitional kernel} if
\begin{itemize}
    \item when $A_2 \in \F_2$ is fixed, $Q(.,A_2)$ is a $\F_1$-measurable function.
    \item when $\omega_1 \in \Omega_1$ is fixed, the set function $Q(\omega_1,.)$ is a measure on $(\Omega_2, \F_2)$
\end{itemize}
In addition if $\forall \omega_1 \in \Omega_1$, $Q(\omega_1, \Omega_2) = 1$ (i.e. $Q(\omega_1,.)$ is a probability measure), then $Q$ is a \textit{(Markov) stochastic kernel}. If we have $\forall \omega_1 \in \Omega_1$, $Q(\omega_1, \Omega_2) \leq 1$, then $Q$ is \textit{substochastic}.
\end{definition}

\begin{remark}
A time-homogeneous Markov chain is characterise by a family of stochastic kernels, which explains why stochastic kernels are named after Markov.
\end{remark}

From this, we can define the notion of regular conditional distribution. Here we assume $\xi$ takes value on a general measurable space $(E,\mathcal{E})$.

\begin{definition}[Regular conditional distribution]
Let $\xi:(\Omega,\F) \to (E,\mathcal{E})$ be a random variable, and let $\G$ be a $\sigma$-algebra of $\Omega$ contained in $\F$. A regular conditional distribution of $\xi$ \textbf{given the $\sigma$-algebra} $\G$ is a stochastic kernel $Q: \Omega \times \mathcal{E} \to [0,\infty]$ such that for almost all $\omega \in \Omega$, we have 
\begin{equation} \label{eq:rcd}
    \forall B \in \mathcal{E}, \quad Q(\omega, B) = [\p(\xi \in B \,|\, \G)](\omega)
\end{equation}
In addition, assume $\eta: (\Omega, \sigma(\eta)) \to (E',\mathcal{E}')$ is a random variable (not necessary equal to $(E,\mathcal{E})$), then the conditional distribution of $\xi$ given the \textbf{random variable $\eta$} is the conditional distribution of $\xi$ given on the $\sigma$-algebra $\sigma(\eta)$.
\end{definition}

\begin{remark} \label{rmk:rcd}
Let $Q:\Omega \times \E$ be a regular conditional distribution of $\xi$ given $\eta$, then for almost all $\omega \in \Omega$, and for all $B \in \mathcal{E}$, we have
\begin{equation}
    Q(\omega,B) = [\p(\xi \in B \,|\, \eta)](\omega)
\end{equation}
When $Q$ is restricted so that the above equality is satisfied, then $Q(.,B)$ is $\eta$ measurable, and hence $Q(\omega,B) = \tilde{Q}(\eta(\omega), B)$ for a unique function $\tilde{Q}: (E',\mathcal{E}') \times \mathcal{E} \to [0,\infty]$, such that $\tilde{Q}(.,B)$ is $\mathcal{E}$ measurable for all $B \in \mathcal{E}$. We will also refer to this function $\tilde{Q}$ when talking about regular conditional distribution of $\xi$ on $\eta$.
\end{remark}

\begin{example}[Continuation of Example \ref{eg:conditional_expectation_joint_rv}] Under same settings, we see that $Q(y,C)$ is a regular conditional distribution of $X$ given $Y$ in the sense as described in remark \ref{rmk:rcd}. Note that when $y$ is fixed, then the measure $Q(y,.)$ is absolutely continuous with density $f_{X|Y}(x|y) := f_{X,Y}(x,y)/f_Y(y)$. The function $f_{X|Y}(x|y)$ is known as the \textit{conditional density} of $X$ given $Y$.
\end{example}

\subsubsection{Existence of regular conditional distribution}

We first assume $(E,\mathcal{E})$ to be $(\R,\B(\R))$ for simplicity. The main difficulty of constructing appropriate regular conditional distribution is that the conditional expectation $[\p(Y \in B \,|\, \G)](\omega)$ is unique up to measure-zero set. Moreover, many properties of probability measures only hold almost surely, e.g. monotonicity and continuity from above/below. Fortunately, we only need the equality \eqref{eq:rcd} to hold for almost all $\omega$. Our plan is therefore to construct the stochastic kernel $Q(\omega,B)$ for almost all "good" $\omega$ when all desired properties hold, and extend this kernel to other "bad" $\omega$. This will work if the set of "bad" $\omega$ is measure zero.

\begin{theorem}[Existence of regular conditional distribution for real-valued random variables] \label{thm:existence_rcd}
Let $\xi:(\Omega,\F,\p) \to (\R,\B(\R))$ be a random variable, then for all $\sigma$-algebra $\G \subseteq \F$ we can construct a regular conditional distribution $\xi$ given $\G$.
\end{theorem}

\begin{hint}
We only need to look at $Q(\omega,B)$ for all $B = (-\infty,r]$ for some $r$, since the half intervals $(-\infty,r]$ generates $\B(\R)$. From chapter 1, we list out the desired property for a version of $F(\omega,r) = [\p[\xi \in (-\infty,r] \,|\, \G]](\omega)$ to be a valid distribution function
\begin{enumerate}
    \item $F(\omega,r) \overset{r \to -\infty}{\to} 0$ and $F(\omega,r) \overset{r \to \infty}{\to} 1$
    \item (Monotonicity) $F(\omega,r) \leq F(\omega,s)$ whenever $r \leq s$, and
    \item (Right continuity) $F(\omega, (-\infty,r_n)) \searrow F(\omega,r)$ whenever $r_n \searrow r$.
\end{enumerate}
So there are four cases for $F(\omega, r)$ not to be a distribution function:
\begin{itemize}
    \item $\omega \in A_{r,s} := \set{\omega \,|\, F(\omega,r) > F(\omega,s)}$ for $r \leq s$.
    \item $\omega \in B_r' := \set{\omega \,|\, \text{there exists a sequence } (r_n) \searrow r \text{ such that } F(\omega,r_n) \not\rightarrow F(\omega,r)}$
    \item $\omega \in C' := \set{\omega \,|\, F(\omega,r) \overset{r\to \infty}{\not\rightarrow} 1}$
    \item $\omega \in D' := \set{\omega \,|\, F(\omega,r) \overset{r\to -\infty}{\not\rightarrow} 0}$
\end{itemize}
If we can show that the "bad" set of $\omega$, $G' = (\cup_{r\leq s} \, A_{r,s}) \cup (\cup_r \, B_r) \cup C \cup D$, has measure zero, then we can define $F(\omega,.)$ as followed:
\begin{equation}
    F(\omega, r) = \begin{cases}
    [\p(\xi \in (-\infty, r] \,|\, \G)](\omega) & \omega \in \Omega \setminus G' \\
    F_0(r) & \omega \in F
    \end{cases}
\end{equation}
where $F_0$ can be any appropriate distribution function. Then for all $\omega$, $F(\omega,r)$ is a distribution function, and for almost all $\omega$, $F(\omega,r) = \p(\xi \in (-\infty,r] \,|\, \G)$. \\

We note from property \ref{prop:conditional_expectation_monotonicity} that $A_{r,s}$ has measure zero for any fixed $r,s$, but this does not imply that $\cup_{r\leq s} A_{r,s}$ has measure zero if we only assume $r,s$ are real, since this is an uncountable union. Moreover, even though we know that for a fixed sequence $(r_n) \searrow r$, we have $F(\omega,r_n) \searrow F(\omega,r)$ almost surely in $\omega$, we don't know whether $B_r$ itself has measure zero, since
\begin{equation}
    B_r' = \bigcup_{(r_n) \text{ sequence} \\ (r_n) \searrow r} \set{\omega \,|\, F(\omega,r_n) \not\to F(\omega,r)}
\end{equation}
which is again an uncountable union of measure zero sets. Fortunately, we can refine our analysis by noting the following elementary facts in analysis: (exercise!)
\begin{itemize}
    \item If $f:\R \to \R$ is a monotonic increasing function, then $f$ is right continuous iff for all $x$, there exists a sequence $(x_n) \searrow x$ such that $f(x_n) \to x$ as $n \to \infty$.
    \item $\Q$ is dense in $\R$. As a result, the half-intervals $(-\infty,a], a\in \Q$ generate $\B(\R)$.
\end{itemize}
As a result, we may refine our analysis by considering $F(\omega,r)$ for all $r \in \Q$, then extend our analysis to $r \in \R$.
\end{hint}

\begin{proof}
We redefine the sets of "bad" $\omega$.
\begin{itemize}
    \item $\omega \in A_{r,s} := \set{\omega \,|\, F(\omega,r) > F(\omega,s)}$ for $r \leq s$
    \item $\omega \in B_r := \set{\omega \,|\, F(\omega, r + 1/n) \not\rightarrow F(\omega,r)}$
    \item $\omega \in C := \set{\omega \,|\, F(\omega,n) \overset{n\to \infty}{\not\rightarrow} 1}, n \in \Z_{\geq 1}$
    \item $\omega \in D := \set{\omega \,|\, F(\omega,-n) \overset{n\to \infty}{\not\rightarrow} 0}, n \in \Z_{\geq 1}$.
\end{itemize}
Define 
\begin{equation}
    G = \bracket{\bigcup_{r\leq s, \\ r,s \in \Q} A_{r,s}} \cup \bracket{\bigcup_{r \in \Q} B_r} \cup C \cup D
\end{equation}
Then $G$ is a countable union of measure-zero sets, and so $G$ is also measure-zero. Hence if we define
\begin{equation}
    F(\omega, z) = \begin{cases}
    \inf\bracket{[\p(\xi \in (-\infty, r] \,|\, \G)](\omega) \,:\, r \in \Q, r > z} & \omega \in \Omega \setminus G \\
    F_0(z) & \omega \in F
    \end{cases}
\end{equation}
then it is a valid distribution function for all $\omega$, and by chapter 1 we can assign a measure $Q(\omega,.)$ to each of the distribution $F(\omega,.)$. Be reminded that $Q(\omega,B)$ is itself a measurable function for all $B \in \B(\R)$, we have therefore constructed a stochastic kernel. \\

We finally check that $Q$ is actually a regular conditional distribution, i.e. for all $A \in \G$,
\begin{equation}
    \E[\chi_A(\omega) Q(\omega, B)] = \E[\chi_{A \cap \set{\xi \in B}}] = \p(A \cap \set{\xi \in B})
\end{equation}
Fix an arbitrary $A \in \G$, then we see that the above equality holds for all $B = (-\infty,r]$ where $r \in \Q$, and hence we can extend the equality to for all $B \in \B(\R)$.
\end{proof}

\begin{remark}
We can extend the above result to any Polish spaces $(E,\mathcal{E})$, for instance $\R^n$ and $C^0([0,1])$ equipped with supremum norm. To do so (theoretically) we note that there is a bijective map $\varphi:(E,\mathcal{E}) \to (\R,\B(\R))$ such that $\varphi$ is $\mathcal{E}$ measurable and $\varphi^{-1}$ is $\B(\R)$ measurable. We can hence construct regular conditional distribution on $\xi:(\Omega,\F) \to (E,\mathcal{E})$ by constructing a regular conditional distribution on $\xi' = \phi \circ \xi: (\Omega,\F) \to (\R,\B(\R))$, denote as $\overline{Q}:\Omega \times \B(\R) \to [0,\infty]$. Then the stochastic kernel $Q(\omega,A) = \overline{Q}(\omega,\phi(A)) : \Omega \times \mathcal{E} \to [0,\infty]$ is indeed a regular conditional distribution of $\xi$.
\end{remark}

\subsubsection{Further Examples}

\begin{example}[Continuation of exercise \ref{ex:conditional_expectation_discrete_rv}]
 Consider random variables $Z_1, Z_2$ on $(\Omega,\F,\p)$ with $Z_1 \sim \Po(\lambda_1)$ and $Z_2 \sim \Po(\lambda_2)$. Assume $p = \lambda_1/(\lambda_1+\lambda_2)$. We have shown that the following regular conditional distribution of $Z_1$ given $Z_1 + Z_2$
\begin{equation}
    \p[Z_1 = k \,|\, Z_1 + Z_2 = n] = {n \choose k} p^k (1-p)^{n-k}
\end{equation}
is actually the probability mass of a binomial distribution $\mathsf{B}(n,p)$. As a result, we may abuse definitions to say that of $Z_1$ given $Z_1 + Z_2$ is "$\mathsf{B}(n,p)$" distributed, written as $Z_1 \,|\, Z_1 + Z_2 \sim \mathsf{B}(n,p)$.
\end{example}

\begin{exercise}[Evaluating conditional expectation from conditional distribution]
Let $Q(\omega,B)$ be a regular conditional distribution of random variable $\xi: (\Omega,\F,\p) \to (E,\mathcal{E})$ given $\sigma$-algebra $\G\subseteq\F$. Assume $h:(E,\mathcal{E}) \to (\R,\B(\R))$ is a $\mathcal{E}$-measurable function such that $\E|h(\xi)| < \infty$. 
\begin{enumerate}
    \item Show that
\begin{equation}
    [\E[h(\xi)\,|\,\G]](\omega) = \int h(x) Q(\omega, dx) 
\end{equation}
where the integral at the RHS is defined so that
\begin{equation}
    \int \chi_B(x) Q(\omega, dx) = \int \chi_B(x) \, dQ(\omega, .) =  Q(\omega,B)
\end{equation}
    \item Define the map $\overline{Q}$ mapping $f$ as a bounded function on $(E,\mathcal{E})$ to a function $\overline{Q}f$ on $(\Omega,\F)$ such that:
    \begin{equation}
        \overline{Q}: f \mapsto \bracket{\omega \mapsto \int f(x) \, Q(\omega, dx)}
    \end{equation}
    Show that $\overline{Q}f$ is bounded, and that $\overline{Q}$ is a contraction int the following sense:
    \begin{equation}
        \sup_{\omega \in \Omega} |\overline{Q}f(\omega)| \leq \sup_{x \in E} |f(x)|
    \end{equation}
\end{enumerate}
\end{exercise}

\begin{hint}
For question 1, one can prove the desired result to $h$ being simple function by using linearity of integrals. Then follow the remaining steps of the four-step proof by utilising appropriate convergence theorems.
\end{hint}

\begin{exercise}[Bayesian Analysis]
\begin{enumerate}
    \item[]
    \item Construct a random vector $(\lambda,N): (\Omega,\F) \to ([0,\infty), \B([0,\infty))) \otimes (\N, 2^\N)$ such that $\lambda \sim \Gamma(\alpha,\beta)$ and $N \,|\, \lambda \sim \Po(\lambda)$.
    \item Show that $\lambda \,|\, N$ follows a Poisson distribution with a suitable parameter as a function of $N$.
\end{enumerate}
\end{exercise}
\end{unexaminable}
\newpage